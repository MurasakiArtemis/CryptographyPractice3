\documentclass[titlepage, 12pt]{article}

\usepackage[utf8]{inputenc}			%Caracteres
\usepackage{amssymb, amsmath, amsfonts, mathtools}		%Matemáticas
\usepackage[bookmarks,hidelinks]{hyperref}	%Links en el ToC
\usepackage[usenames,dvipsnames]{color}		%Colores
\usepackage{listings}				%Código
\usepackage{url}				%Links web
\usepackage[hypcap]{caption}	%Imagenes
\usepackage{float}				%Imágenes

%Permite el cuarto nivel de subsecciones
\usepackage{titlesec}
\setcounter{secnumdepth}{5}
\titleformat{\paragraph}{\normalfont\normalsize\bfseries}{\theparagraph}{1em}{}
\titlespacing*{\paragraph}{0pt}{3.25ex plus 1ex minus .2ex}{1.5ex plus .2ex}
%%%

%Ajustar márgenes
\topmargin=-0.45in
\evensidemargin=0in
\oddsidemargin=0in
\textwidth=6.5in
\textheight=9.0in
\headsep=0.25in

\numberwithin{equation}{section}%Número de ecuaciones (#Sección, #Ecuación)
\numberwithin{figure}{section}%Número de imágenes (#Sección, #Imagen)
\numberwithin{table}{section}%Número de imágenes (#Sección, #Tabla)

\lstdefinestyle{customcpp}{
	language=C++,
	frame=single, % Single frame around code
	basicstyle=\small\ttfamily, % Use small true type font
	keywordstyle=[1]\color{Blue}\bf, % C++ functions bold and blue
	keywordstyle=[2]\color{Purple}, % C++ function arguments purple
	keywordstyle=[3]\color{Blue}\underbar, % Custom functions underlined and blue
	identifierstyle=, % Nothing special about identifiers                                         
	commentstyle=\usefont{T1}{pcr}{m}{sl}\color[rgb]{0.0,0.4,0.0}\small, % Comments small dark green courier font
	stringstyle=\color{Purple}, % Strings are purple
	showstringspaces=false, % Don't put marks in string spaces
	tabsize=5, % 5 spaces per tab
	% Put standard C++ functions not included in the default language here
	morekeywords={std,string,cout,endl,ifstream,ofstream,ios},
	% Put C++ function parameters here
	morekeywords=[2]{iostream,fstream},
	% Put user defined functions here
	morekeywords=[3]{},
	morecomment=[l][\color{Blue}]{...}, % Line continuation (...) like blue comment
	numbers=left, % Line numbers on left
	firstnumber=1, % Line numbers start with line 1
	numberstyle=\tiny\color{Blue}, % Line numbers are blue and small
	stepnumber=5, % Line numbers go in steps of 5
	extendedchars=true
	inputencoding=utf8,
	literate={á}{{\'a}}1 {é}{{\'e}}1 {í}{{\'i}}1 {ó}{{\'o}}1 {ú}{{\'u}}1 {ñ}{{\~n}}1,
}

%Ajustar código C
\lstdefinestyle{customc}{
	language=C,
	frame=single, % Single frame around code
	basicstyle=\small\ttfamily, % Use small true type font
	keywordstyle=[1]\color{Blue}\bf, % C functions bold and blue
	keywordstyle=[2]\color{Purple}, % C function arguments purple
	keywordstyle=[3]\color{Blue}\underbar, % Custom functions underlined and blue
	identifierstyle=, % Nothing special about identifiers                                         
	commentstyle=\usefont{T1}{pcr}{m}{sl}\color[rgb]{0.0,0.4,0.0}\small, % Comments small dark green courier font
	stringstyle=\color{Purple}, % Strings are purple
	showstringspaces=false, % Don't put marks in string spaces
	tabsize=2, % 2 spaces per tab
	% Put standard C functions not included in the default language here
	morekeywords={},
	% Put C function parameters here
	morekeywords=[2]{},
	% Put user defined functions here
	morekeywords=[3]{},
	morecomment=[l][\color{Blue}]{...}, % Line continuation (...) like blue comment
	numbers=left, % Line numbers on left
	firstnumber=1, % Line numbers start with line 1
	numberstyle=\tiny\color{Blue}, % Line numbers are blue and small
	stepnumber=5, % Line numbers go in steps of 5
	extendedchars=true,
	inputencoding=utf8,
	literate={á}{{\'a}}1 {é}{{\'e}}1 {í}{{\'i}}1 {ó}{{\'o}}1 {ú}{{\'u}}1 {ñ}{{\~n}}1,
}

\title{Report 1}
\author{Eron Romero Argumedo\\Erwin Hernandez Garcia\\3CV1}

\newcommand{\imagen}[4][]{
	\begin{figure}[H]
		\centering
		\includegraphics[#1]{#2}
		\caption{#3}
		#4
	\end{figure}
}

\newcommand{\codescript}[4][]{
	\begin{itemize}
		\item[]\lstinputlisting[caption=#4,label=#3,style=custom#2, #1]{#3.#2}
	\end{itemize}
}

\newcommand{\delimitCodeScript}[6][]{
	\begin{itemize}
		\item[]\lstinputlisting[lastline=#6,firstline=#5, caption=#4,label=#3#5#6,style=custom#2, #1]{#3.#2}
	\end{itemize}	
}

\begin{document}
	\maketitle
	\pagenumbering{roman}
	\tableofcontents
	\listoffigures
	\listoftables
	\newpage
	\pagenumbering{arabic}
	\section{Theory}
		\subsection{How does the extended Euclidean algorithm works}
		If $gcd(a,b)=d$, then exist integers u and v such that $d = u*a + v*b$; the gcd of two numbers can be expressed as a linear combination of those two numbers with integer coefficients. \bigskip
		
		The Extended Euclidean Algorithm can determinate the values of u and v and it is a direct application of the Euclidean Algorithm. It can also calculate the inverse of a number on modular arithmetic.\cite{ExtendedEuclidean} \bigskip
		
		\subsubsection{Example}
		\paragraph*{Euclidean Algorithm}
		We want to find $u$ and $v$ such that $d = u*a + v*b$ where $d=gcd(135, 47)$.
		First we need to calculate the value of $d$, so:
			\begin{equation}\label{FirstStepEuclidean}
			135 = (47)*(2) + 41
			\end{equation}
			\begin{equation}\label{SecondStepEuclidean}
			47 = (41)*(1) + 6
			\end{equation}
			\begin{equation}\label{ThirdStepEuclidean}
			41 = (6)*(6) + 5
			\end{equation}
			\begin{equation}\label{FourtStepEuclidean}
			6 = (5)*(1) + 1
			\end{equation}
			\begin{equation}\label{FifthStepEuclidean}
			5 = (1)*(5) + 0
			\end{equation}
		\paragraph*{Extended Euclidean Algorithm}
		We can notice that we have the form $a = b*q + r$, so we can change to $r = a - b*q$ and we can reorder the above equations:
		\begin{enumerate}
			\item First element
			\begin{equation}
				41 = (1)*(135) - (2)*(47)
			\end{equation}
			\item Second element
			\begin{equation}
			\begin{split}
			6 &= 47 - (41)*(1)\\
			&= 47 + (41)*(-1)\\
			&= 47 + (135 - (47)*(2))*(-1)\\
			&= 47 + (135)(-1) + (47)*(2)\\
			&= (-1)*(135) + (3)*(47)
			\end{split}
			\end{equation}
			\item Third element
			\begin{equation}
				\begin{split}
				5 &= 41 - (6)*(6) \\
				&= 135 - (47)*(2) - ((47)(3) + (135)(-1))*(6) \\
				&= 135 - (47)*(2) - (47)*(18) - (135)*(-6)\\
				&= (7)*(135) + (-20)*(47)  \\
				\end{split}
			\end{equation}
			\item Fourth element
			\begin{equation}
				\begin{split}
				1 &= 6 - (5)*(1)\\ 
				&= (-1)*(135) + (3)*(47) - ((7)*(135) + (-20)*(47))*(1)\\ 
				&= (-1)*(135) + (3)*(47) - (7)*(135) - (-20)*(47)\\ 
				&= (-8)*(135) + (23)*(47)\\
				\end{split} 
			\end{equation}
		\end{enumerate}
		
		We only need to do the algorithm until we find the equation which has a remainder of 0. \bigskip
		
		So now we know that $d=1$, $u=-8$ and $v=47$ because\newline
		\begin{equation}
			1 = (-8)*(135) + (23)*(47)
		\end{equation}
		\subsection{How to make a known-plaintext attack to the Hill cipher, if we know that the key is 2 x 2 matrix}
		
		If we do not know the key but we know that this key is a 2x2 matrix then we need to know $n^{2}$ plaint text (where n corresponds to the matrix's dimension, in this case $K_{nxn}$).\cite{hill} \bigskip
		
		If we know the cipher text and we know $n^{2}$ plain text, in our case $2^{2}=4$, the we can do the following. \bigskip
		
		\begin{enumerate}
		\item Lets suppose that we have the following cipher text:
		\begin{center}
			LLZCKBFLMHD
		\end{center} 
		
		Which correspond to the following numbers in the English alphabet:
		\begin{center}
			11 11 25 2 10 1 5 11 12 7 3 11
		\end{center} 
		\item Supposing we know that the plain text contains the following letters \textbf{LLOF}.
		\item To find the key, first we need to tie the cipher text with the plain thex that we have:
		
		\begin{table}[H]
			\centering
			\caption{Possible combinations of ciphertext to plaintext}
			\begin{tabular}{|c|c|c|c|c|c|c|c|c|c|c|c|}
				\hline
				L & L & Z & C & K & B & F & L & M & H & D & L \\ \hline
				L & L & O & F &   &   &   &   &   &   &   &  \\ \hline
				  & L & L & O & F &   &   &   &   &   &   &  \\ \hline
				  &   & L & L & O & F &   &   &   &   &   &  \\ \hline
				  &   &   &   &   & . & . & . & . &   &   &  \\ \hline
				  &   &   &   &   &   &   &   & L & L & O & F \\ \hline
			\end{tabular}
			
		\end{table}
		
		We need to obtain a linear system for each combination and then we need to solve each of them until we find a key which give us a coherent message.
		The formula to decipher is:
		\begin{equation}
			M = K^{-1}C
		\end{equation}
		Therefore to obtain the key we multiply the equation by $C^{-1}$.
		\begin{equation}
			MC^{-1} = K^{-1}
		\end{equation}
		\item We take the first case LLZC which corresponds to LLOF.
		
		\begin{equation}
			\text{LL} = 
			\begin{pmatrix}
			k_{0,0} & k_{0,1} \\
			k_{1,0} & k_{1,0}
			\end{pmatrix}^{-1}
			\begin{pmatrix}
				11 \\
				11
			\end{pmatrix}
			\bmod{26} = 
			\begin{pmatrix}
			11 \\
			11 
			\end{pmatrix}
			\bmod{26} = \text{LL}
		\end{equation}
		
		\begin{equation}
			\text{ZC} = 
			\begin{pmatrix}
			k_{0,0} & k_{0,1} \\
			k_{1,0} & k_{1,0}
			\end{pmatrix}^{-1}
			\begin{pmatrix}
				25 \\
				2
			\end{pmatrix}
			\bmod{26} = 
			\begin{pmatrix}
				14 \\
				5
			\end{pmatrix}
			\bmod{26} = \text{OF}
		\end{equation}
		
		We obtain:
		
		\begin{equation}
			\begin{pmatrix}
			k_{0,0} & k_{0,1} \\
			k_{1,0} & k_{1,0}
			\end{pmatrix}^{-1}
			\begin{pmatrix}
			11 & 25 \\
			11 & 2
			\end{pmatrix}
			\bmod{26} = 
			\begin{pmatrix}
			11 & 14 \\
			11 & 5 
			\end{pmatrix}
			\bmod{26}
		\end{equation}
		
		\item Now we need to find the value of $K^{-1}$, determine by:
		
		\begin{equation}
			K^{-1} = 
			\begin{pmatrix}
			11 & 14 \\
			11 & 5
			\end{pmatrix}
			\begin{pmatrix}
			11 & 25 \\
			11 & 2
			\end{pmatrix}^{-1}
			\bmod{26}
		\end{equation}
		
		\item Let $ A = \left(\begin{smallmatrix}
		11 & 25 \\
		11 & 2
		\end{smallmatrix}\right)$, then we obtain $A^{-1}$ 
		
		\begin{equation}
			A^{-1} = \begin{pmatrix}
			4 & 15 \\
			17 & 9
			\end{pmatrix}\bmod{26}
		\end{equation}
		
		\item Now we can get, the value of $K^{-1}$:
		
		\begin{equation}
			K^{-1} = 
			\begin{pmatrix}
			11 & 14 \\
			11 & 5
			\end{pmatrix}
			\begin{pmatrix}
			4 & 15 \\
			17 & 9
			\end{pmatrix}\bmod{26} = 
			\begin{pmatrix}
			22 & 5 \\
			25 & 2
			\end{pmatrix}\bmod{26}			
		\end{equation}
		
		Now we try to decipher the message with that key:
		
		\begin{equation*}
			\begin{pmatrix}
			22 & 5 \\
			25 & 2
			\end{pmatrix}
			\begin{pmatrix}
			11 \\
			11
			\end{pmatrix}\bmod{26} = 
			\begin{pmatrix}
			11 \\
			11
			\end{pmatrix}\bmod{26}
		\end{equation*}
		
		\begin{equation*}
			\begin{pmatrix}
			22 & 5 \\
			25 & 2
			\end{pmatrix}
			\begin{pmatrix}
			25 \\
			2
			\end{pmatrix}\bmod{26} = 
			\begin{pmatrix}
			14 \\
			5
			\end{pmatrix}\bmod{26}
		\end{equation*}
		
		\begin{equation*}
			\begin{pmatrix}
			22 & 5 \\
			25 & 2
			\end{pmatrix}
			\begin{pmatrix}
			10 \\
			1
			\end{pmatrix}\bmod{26} = 
			\begin{pmatrix}
			17 \\
			18
			\end{pmatrix}\bmod{26}
		\end{equation*}
		
		\begin{equation*}
			\begin{pmatrix}
			22 & 5 \\
			25 & 2
			\end{pmatrix}
			\begin{pmatrix}
			5 \\
			11
			\end{pmatrix}\bmod{26} = 
			\begin{pmatrix}
			9 \\
			17
			\end{pmatrix}\bmod{26}
		\end{equation*}
		
		\begin{equation*}
			\begin{pmatrix}
			22 & 5 \\
			25 & 2
			\end{pmatrix}
			\begin{pmatrix}
			12 \\
			7
			\end{pmatrix}\bmod{26} = 
			\begin{pmatrix}
			13 \\
			2
			\end{pmatrix}\bmod{26}
		\end{equation*}
		
		\begin{equation*}
			\begin{pmatrix}
			22 & 5 \\
			25 & 2
			\end{pmatrix}
			\begin{pmatrix}
			3 \\
			11
			\end{pmatrix}\bmod{26} = 
			\begin{pmatrix}
			17 \\
			19
			\end{pmatrix}\bmod{26}
		\end{equation*}		
		
		The message that we obtain is the following:
		
		\begin{table}[H]
			\centering
			\caption{Deciphered message with an incorrect key}
			\begin{tabular}{cccccccccccc}
				11&11&14&5&17&18&9&17&13&2&17&19\\
				L&L&O&F&R&S&J&R&N&C&R&T		
			\end{tabular}
		\end{table}
		
		\item As we can see, the plain text LLOF is in our message but our message has not coherence. So we need to keep trying until we find a message with coherence. This occur when we try with ZCKB which correspond with LLOF.\bigskip
		
		We find that in this case $K^{-1} = 
		\left(\begin{smallmatrix}
		7 & 22 \\
		21 & 3
		\end{smallmatrix}\right)$.\bigskip
		
		Then we try to decipher the message as we did before:
		
		\begin{equation*}
			\begin{pmatrix}
			7 & 22 \\
			21 & 3
			\end{pmatrix}
			\begin{pmatrix}
			11 \\
			11
			\end{pmatrix}\bmod{26} = 
			\begin{pmatrix}
			7 \\
			4
			\end{pmatrix}\bmod{26}
		\end{equation*}
		
		\begin{equation*}
			\begin{pmatrix}
			7 & 22 \\
			21 & 3
			\end{pmatrix}
			\begin{pmatrix}
			25 \\
			2
			\end{pmatrix}\bmod{26} = 
			\begin{pmatrix}
			11 \\
			11
			\end{pmatrix}\bmod{26}
		\end{equation*}
		
		\begin{equation*}
			\begin{pmatrix}
			7 & 22 \\
			21 & 3
			\end{pmatrix}
			\begin{pmatrix}
			10 \\
			1
			\end{pmatrix}\bmod{26} = 
			\begin{pmatrix}
			14 \\
			5
			\end{pmatrix}\bmod{26}
		\end{equation*}
		
		\begin{equation*}
			\begin{pmatrix}
			7 & 22 \\
			21 & 3
			\end{pmatrix}
			\begin{pmatrix}
			5 \\
			11
			\end{pmatrix}\bmod{26} = 
			\begin{pmatrix}
			17 \\
			8
			\end{pmatrix}\bmod{26}
		\end{equation*}
		
		\begin{equation*}
			\begin{pmatrix}
			7 & 22 \\
			21 & 3
			\end{pmatrix}
			\begin{pmatrix}
			12 \\
			7
			\end{pmatrix}\bmod{26} = 
			\begin{pmatrix}
			4 \\
			13
			\end{pmatrix}\bmod{26}
		\end{equation*}
		
		\begin{equation*}
			\begin{pmatrix}
			7 & 22 \\
			21 & 3
			\end{pmatrix}
			\begin{pmatrix}
			3 \\
			11
			\end{pmatrix}\bmod{26} = 
			\begin{pmatrix}
			3 \\
			18
			\end{pmatrix}\bmod{26}
		\end{equation*}		
		
		The message that we obtain is the following:
		
		\begin{table}[H]
			\centering
			\caption{Deciphered message}
			\begin{tabular}{cccccccccccc}
			7&4&11&11&14&5&17&8&4&13&3&18\\
			H&E&L&L&O&F&R&I&E&N&D&S
			\end{tabular}
		\end{table}
		\end{enumerate}
	\section{Functions}
		\delimitCodeScript{cpp}{Exercise1}{Implementation of the extended euclidean algorithm}{18}{33}
		\delimitCodeScript{c}{Exercise2}{Implementation of the given algorithm}{25}{54}
		\delimitCodeScript{c}{Exercise3}{Implementation of the given algorithm}{22}{53}
		On (\ref{Exercise32253}) we deleted the variables that gave the value for n. This was done because those coefficients are neither necessary to obtain the inverse nor useful information.
		\delimitCodeScript{cpp}{CipherCracking}{The steps necessary to obtain the key}{50}{59}
		\delimitCodeScript{cpp}{Functions}{The function to calculate the inverse of a matrix}{43}{65}
		\delimitCodeScript{cpp}{Functions}{The function to multiply two square matrices}{84}{100}
		\delimitCodeScript{cpp}{Functions}{The function that implements the Gauss-Jordan algorithm}{102}{132}
		\imagen[width=\linewidth]{Picture1}{The first three exercises}{\label{First3}}
		\imagen[width=\linewidth]{Picture2}{The program that breaks the cipher}{\label{Break}}
	\bibliographystyle{IEEEtran}
	\bibliography{Bibliography}
\end{document}