\documentclass[titlepage, 12pt]{article}

\usepackage[utf8]{inputenc}			%Caracteres
\usepackage{amssymb, amsmath, amsfonts, mathtools}		%Matemáticas
\usepackage[bookmarks,hidelinks]{hyperref}	%Links en el ToC
\usepackage[usenames,dvipsnames]{color}		%Colores
\usepackage{listings}				%Código
\usepackage{url}				%Links web
\usepackage[hypcap]{caption}	%Imagenes
\usepackage{float}				%Imágenes

%Ajustar márgenes
\topmargin=-0.45in
\evensidemargin=0in
\oddsidemargin=0in
\textwidth=6.5in
\textheight=9.0in
\headsep=0.25in

\numberwithin{equation}{section}%Número de ecuaciones (#Sección, #Ecuación)
\numberwithin{figure}{section}%Número de imágenes (#Sección, #Imagen)
\numberwithin{table}{section}%Número de imágenes (#Sección, #Tabla)

\lstdefinestyle{customcpp}{
	language=C++,
	frame=single, % Single frame around code
	basicstyle=\small\ttfamily, % Use small true type font
	keywordstyle=[1]\color{Blue}\bf, % C++ functions bold and blue
	keywordstyle=[2]\color{Purple}, % C++ function arguments purple
	keywordstyle=[3]\color{Blue}\underbar, % Custom functions underlined and blue
	identifierstyle=, % Nothing special about identifiers                                         
	commentstyle=\usefont{T1}{pcr}{m}{sl}\color[rgb]{0.0,0.4,0.0}\small, % Comments small dark green courier font
	stringstyle=\color{Purple}, % Strings are purple
	showstringspaces=false, % Don't put marks in string spaces
	tabsize=5, % 5 spaces per tab
	% Put standard C++ functions not included in the default language here
	morekeywords={std,string,cout,endl,ifstream,ofstream,ios},
	% Put C++ function parameters here
	morekeywords=[2]{iostream,fstream},
	% Put user defined functions here
	morekeywords=[3]{},
	morecomment=[l][\color{Blue}]{...}, % Line continuation (...) like blue comment
	numbers=left, % Line numbers on left
	firstnumber=1, % Line numbers start with line 1
	numberstyle=\tiny\color{Blue}, % Line numbers are blue and small
	stepnumber=5, % Line numbers go in steps of 5
	extendedchars=true
	inputencoding=utf8,
	literate={á}{{\'a}}1 {é}{{\'e}}1 {í}{{\'i}}1 {ó}{{\'o}}1 {ú}{{\'u}}1 {ñ}{{\~n}}1,
}

\title{Operation Modes}
\author{Eron Romero Argumedo\\Erwin Hernandez Garcia\\3CV1}
\date{September 20, 2016}

\newcommand{\imagen}[4][]{
	\begin{figure}[H]
		\centering
		\includegraphics[#1]{#2}
		\caption{#3}
		#4
	\end{figure}
}

\newcommand{\codescript}[4][]{
	\begin{itemize}
		\item[]\lstinputlisting[caption=#4,label=#3,style=custom#2, #1]{#3.#2}
	\end{itemize}
}

\newcommand{\delimitCodeScript}[6][]{
	\begin{itemize}
		\item[]\lstinputlisting[lastline=#6,firstline=#5, caption=#4,label=#3#5#6,style=custom#2, #1]{#3.#2}
	\end{itemize}	
}

\begin{document}
	\maketitle
	\pagenumbering{roman}
	\tableofcontents
	\listoffigures
	\newpage
	\pagenumbering{arabic}
	\section{Theory}
		\subsection{How do permutations work}
		A permutation in cryptography consists in a rearrangement of the elements in a specified order, where that order is both the key and the function to encrypt.\cite{PermutationCiphers}
		\subsection{How does CBC work}
		CBC is an operation mode that uses a initialization vector (IV) and realizes an XOR operation with the plaintext, whose result is then encrypted to obtain the cyphertext. In the following rounds the IV is substituted with the previous ciphertext. For decryption it uses the ciphertext which is decrypted and then an XOR is done with the IV. For the following rounds the IV is replaced with the previous ciphertext.\cite{BlockCipher}
		\subsection{How does CTR work}
		The CTR operation mode consists in calculating an XOR between the plaintext and the encryption of the IV for encryption. For decryption it calculates the XOR between the ciphertext and the encryption of the IV.
		On each round the IV is incremented by 1.\cite{BlockCipher}
	\section{Functions}
		\subsection{ECB}
		\delimitCodeScript{cpp}{Functions}{Key generation}{18}{30}
		The key generation creates a permutation and its inverse, then save them in a file with the specified name.
		\delimitCodeScript{cpp}{Functions}{Main function}{40}{95}
		
		This function is used to partition the plain or cipher text and to call the specified mode of operation.
		
		The IV is created randomly in the case of encryption and extracted from the beginning of the ciphertext if decryption.
		\subsection{CBC}
		\delimitCodeScript{cpp}{Functions}{CBC operation mode}{97}{113}
		The function to cipher is called each round, if the encryption is solicited, then the result array starts as the IV which is randomly generated, on the other case, the aux array is the one that contains the IV.
		\subsection{CTR}
		\delimitCodeScript{cpp}{Functions}{CTR operation mode}{114}{121}
		If the CTR mode is specified, the key used is for encryption regardless the option used. In any other case, for encryption, the second key (inverse permutation) is extracted.
	\section{Screenshots}
	\subsection{Usage}
		\imagen[width=\linewidth]{Picture1}{Usage of the program}{\label{Usage}}
	\subsection{Results}
		\imagen[width=\linewidth]{Picture2}{Original plaintext and the plaintext obtained after encryption and decryption}{\label{Plaintext}}
		\imagen[width=\linewidth]{Picture3}{Ciphertext obtained using the different modes of operation}{\label{Ciphertext}}
	\bibliographystyle{IEEEtran}
	\bibliography{Bibliography}
\end{document}